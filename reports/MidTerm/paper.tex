% TEMPLATE for Usenix papers, specifically to meet requirements of
%  USENIX '05
% originally a template for producing IEEE-format articles using LaTeX.
%   written by Matthew Ward, CS Department, Worcester Polytechnic Institute.
% adapted by David Beazley for his excellent SWIG paper in Proceedings,
%   Tcl 96
% turned into a smartass generic template by De Clarke, with thanks to
%   both the above pioneers
% use at your own risk.  Complaints to /dev/null.
% make it two column with no page numbering, default is 10 point

% Munged by Fred Douglis <douglis@research.att.com> 10/97 to separate
% the .sty file from the LaTeX source template, so that people can
% more easily include the .sty file into an existing document.  Also
% changed to more closely follow the style guidelines as represented
% by the Word sample file. 

% Note that since 2010, USENIX does not require endnotes. If you want
% foot of page notes, don't include the endnotes package in the 
% usepackage command, below.

% This version uses the latex2e styles, not the very ancient 2.09 stuff.
%\documentclass[letterpaper,twocolumn,10pt]{article}
%\usepackage{usenix,epsfig,endnotes}
\documentclass{ns-article-compact}
\usepackage{amsmath}
\usepackage{amsthm}
\usepackage{algorithm}
%\usepackage{algorithmic}
\usepackage{algpseudocode}
\usepackage{caption}
\usepackage[utf8]{inputenc}

\usepackage{wasysym}
\usepackage{xspace}
\usepackage{balance}

\usepackage{color}
\usepackage{mathrsfs}
\usepackage{multirow}
\usepackage{array}
\usepackage{verbatim}
\usepackage{graphicx,subcaption}
\usepackage{comment}
\usepackage{epsfig}
\usepackage[hyphens]{url}
\usepackage{hyperref}
\usepackage{tabularx}

\newcolumntype{Y}{>{\centering\arraybackslash}X}

\begin{document}

%don't want date printed
\date{}

\newcommand{\squishenum}{
   \begin{enumerate}
    { \setlength{\itemsep}{0pt}      \setlength{\parsep}{1pt}
      \setlength{\topsep}{1pt}       \setlength{\partopsep}{0pt}
      \setlength{\leftmargin}{1.0em} \setlength{\labelwidth}{1em}
      \setlength{\labelsep}{0.5em} } }

\newcommand{\squishlist}{
   \begin{list}{$\bullet$}
    { \setlength{\itemsep}{0pt}      \setlength{\parsep}{3pt}
      \setlength{\topsep}{3pt}       \setlength{\partopsep}{0pt}
      \setlength{\leftmargin}{1.0em} \setlength{\labelwidth}{1em}
      \setlength{\labelsep}{0.5em} } }

\newcommand{\squishlisttwo}{
   \begin{list}{$\bullet$}
    { \setlength{\itemsep}{0pt}    \setlength{\parsep}{0pt}
      \setlength{\topsep}{0pt}     \setlength{\partopsep}{0pt}
      \setlength{\leftmargin}{1em} \setlength{\labelwidth}{0.5em}
      \setlength{\labelsep}{0.5em} } }

\newcommand{\squishend}{
    \end{list}  }

\newcommand{\squishenumend}{
    \end{enumerate}  }

\newcommand{\tightcaption}[1]{\vspace{-12pt}\caption{{\bf \small #1}}
\vspace{-7pt}
}

\newcommand{\eat}[1]{}

\urlstyle{rm}

%space saving macros
\newcommand{\ispace}{\vspace{-0.08in}}
\renewcommand{\ispace}{}
\newcommand{\jspace}{\vspace{-0.06in}}
\renewcommand{\jspace}{}

\newcommand{\willgo}[1]{}
\newcommand{\TBD}[1]{[{\bf{TBD:}} #1]}
\newcommand{\tbd}[1]{[[[{\bf{TBD:}} #1]]]}
\newcommand{\todo}[1]{{\color{red} #1}}
%\newcommand{\todo}[1]{}
\newcommand{\rmv}[1]{\footnote{{\bf{CUT:}} #1}}
\newcommand{\shorten}[1]{#1}

\newcommand{\mypara}[1]{
%\smallskip
\noindent
{\bf {#1}}~}
%\renewcommand{\paragraph}[1]{{\bf #1}}
%\setlength{\textfloatsep}{10pt}
%\setlength{\floatsep}{10pt}
\newcommand{\myred}[1]{{\color{red} {#1}}}
\newcommand{\mycomment}[1]{\textit{\color{blue} {#1}}}

%\setlength{\tabcolsep}{4pt}
\newcommand{\paraspace}{\vspace{0.02in}}
\newcommand{\parab}[1]{\paraspace\noindent{\bf #1} }
\newcommand{\parae}[1]{\paraspace\noindent{\em #1} }
\newcommand{\parabe}[1]{\paraspace\noindent{\bf \em #1} }

\newcommand{\system}{{\sc Pando}\xspace}
\newcommand{\pman}{Placement Manager\xspace}

%make title bold and 14 pt font (Latex default is non-bold, 16 pt)
\def\papertitle{ Memoization of Browser Computations for a Faster Mobile Web}

\title{\papertitle}
\author{
  {\rm Ayush Goel}
  \and
  {\rm Matthew Furlong}
  \and
  {\rm HyunJong (Joseph) Lee}
}
\maketitle

% Use the following at camera-ready time to suppress page numbers.
% Comment it out when you first submit the paper for review.
\thispagestyle{empty}

\section{Introduction}
\label{sec:intro}

Page Load Time (PLT) of a website is a key performance metric that
significantly impacts user-experience, as pointed out by many recent studies
from both academia and industry~\cite{bhatti2000integrating, bouch2000quality}.
User-experience in web browsing is directly correlated to companies' revenues:
Amazon shows that reducing 100ms in PLT results in 1 percent revenue increase
and Shopzilla reports that improving PLT from 6 to 1.2 seconds increased their
revenue by 12 percent~\cite{url3}. 

There have been many prior works in improving the PLT of mobile devices,
ranging from offloading computation and network tasks to
proxies~\cite{netravali2015mahimahi, sivakumar2014parcel, wang2014speedy} to
re-prioritising requests at client side by letting the client itself discover
all resources on a page~\cite{butkiewicz2015klotski, netravali2016polaris}.  It
is worthwhile to note that existing works attempt to surrogate web-tasks of
mobile devices from resource-rich server
environment~\cite{ruamviboonsuk2017vroom}.
 
The end-to-end PLT for many webpages is far from ideal: an order
of tens of seconds on mobile devices~\cite{wang2013demystifying} and
on the order of seconds for stationary desktops.  Many existing solutions
often require server-side modification, which strongly discourages
content providers to use these new solutions. 
% Joseph: why doing at server-side is bad?
However, a client side solution would be agnostic of the content
provider/server that is used to render the web pages and
can therefore optimize page load times for all web pages alike. 
Most of the existing work on client side optimizations focuses on efficient ways to
optimise web cache~\cite{wang2014much}.  
Prior work shows that computation latency is the driving factor behind slow page load time for mobile devices, as compared to newtork latency.~\cite{vesuna2016caching}.

In this work, we propose a novel PLT optimization technique that caches output
from previous code execution of a webpage (e.g., Javascript, inline
HTML, css) on mobile devices to reduce user-perceived PLT at the small cost
of increased storage requirements.
Prior techniques have gone as far as caching the compiled code, either
on the client side or on the server side, to save on the compilation
time when the web content remains unchanged~\cite{wang2014much}. We
take this a step further, and cache the output of the execution of all
the code on a web page. ( Note the use the word code, to
distinguish it from other components of a webpage which include layout
and data). Recent work has shown that most of the webpages remain
unchanged over a large period of time. For content-rich pages, the
amount of updates vary across Web pages. In the best (worst) case,
20\% (75\%) of the HTML page is changed over a month. Most changes are
made to data (e.g., links to images, titles) while little change is made to
the layout and code~\cite{wang2014much}.  This implies that most of the
code output could be reused, essentially eliminating code execution
time from the critical path of a web page load. This would bring down
the entire page load time to the time taken in rendering and painting the
layout. 
Caching the computation as a technique to optimize the execution time has
already been explored at a data center level~\cite{gunda2010nectar} and it has
shown tremendous improvement with more than 35\% of jobs benefiting from caching.
We are trying to apply a similar technique on the mobile client's browser.
\section{Motivation}
\label{sec:motivation}


Major web browsers like Chrome, Firefox, and Safari have recenlty invested a lot of resources,
time and energy into improving web performance on mobile devices, specifically by targeting 
the network usage. However, the network now comprises less than 30\% \cite{njait2016www} of the total critical
path for an average page load on a mobile device. This includes caching almost 95\% of the
resources that are fetched from the server \cite{vesuna2016caching}, dns presolution, dns caching, tcp reconnect etc.
Chrome released a paper last year showing how improved caching algorithms, despite having 
significant improvements on the desktop, don't have the same proportionate imrpovements 
on mobile devices. This is primarily attributed to the fact that computation comprises more than 65\%
of the critical path during a page load. This illustrates the need to further optimize the compuatation 
time. 

During the Chrome dev summit this year, their team announced the latest improvements they have 
made in their browser to improve the page load time. Interstingly, most of their work focuses on
improving the compilation and parsing time by introducing compile and parser cache. 
Recent studies \cite{url4} still report that the median page load time for a mobile website 
is about 14 seconds. Research \cite{url4} shows that a user will only wait for 3 seconds 
before abandoning a web site if it shows no response at all. A lot of prior work \cite {njait2016www}
has been done to compare the page load times on mobile vs desktop, and recent results
from 2016 claim that despite the increasing compute resources in mobile devices,
the computation time on mobile is significanlty higher than their desktop counterparts. 
 Our experiments
on the most popular news and sports websites on the latest mobile hardware and the latest 
Chrome verison reveal that despite these recent efforts, scripting still takes significantly more
time ,as compared to the other components of the total computation time. We break down computation
into four categories: scripting, loading, paitnint, and rendering.
and observe that scripting essentially takes more than 70\% of the total
computation time which is more than all the other categories combined (Figure 3). 
This makes it all the more important to do an in-depth analysis of the computation time to clearly
understand where exactly this time is being spent. 

\section{Design}
\label{sec:design}

\begin{figure}[!h]
\begin{subfigure}[h]{0.5\textwidth}
\centering
\includegraphics[width=\linewidth]{figs/comp_net.png}
\caption{Runtime information on mobile devices}
\label{fig:mobile-runtime}
\end{subfigure}
\begin{subfigure}[h]{0.5\textwidth}
\centering
\includegraphics[width=\linewidth]{figs/comp_net_desk.png}
\caption{Runtime information on desktops}
\label{fig:mobile-runtime}
\end{subfigure}
\caption{{\color{red}Figure 1 captioon placeholder}}
\end{figure}
We propose a new technique to imrpove the page load time by reducing
the javscript time.  In order to do this we are trying to build a new
caching framework for the modern web browsers specifically, Chrome
since it accounts for about 50\% of the market share in terms of
browser usage. Our caching framework will store the javascript
execution result. This can mean a lot of things due to the dynamic
nature javascript. Most of the times it is supposed to be the return
value of the javascript functions. At other times it can be a modified
DOM structure or just some intermediate result which is further
processed by other javascript, later down the execution timeline.  The
expiry of this javascript exectution cache is supposed to be same as
the expire of the javascript source cache.  There are a lot of caveats
to this approach, and in our work we try to essentially explore all of
these.  The biggest challenge with a new caching framework are the
actual modifications to the current browser's code in order to
evaluate the efficacy of our caching frameowork. Since a lot of
browsers already implement caching at the javascript runtime level,
such as compile and parses cache, a lot of this architecure can be
borrowed for the execution cache as well.  Another possible challenge
can be the memory overhead. Most of the popular websites which spend
about 70\% of their time on javascript execution, run about 1000s of
javascript functions. Saving the output of all of these functions can
add an extra overhead on the current browsers. 

\section{Progress}
\label{sec:Progress}


\begin{figure}[t!]
\centering
\includegraphics[width=0.99\columnwidth, scale=2.0]{figs/comp_1.png}
\tightcaption{Breakdown of computation on pixel 2}
\label{fig:act_p2}
\end{figure}

As a preliminary step, we first established a corpus of the top
75 news and sports websites to cater to the most popular and compute intensive websites. 
These websites were gathered from the Alexa top website list.
We ran all our experiments on a Google Pixel 2 with Chrome version 61. We leveraged chrome developer tools in order
to capture runtime traces for both networking and computation. We then analyzed these runtime
traces to draw insight into the critical path of the website, the total computation time vs the total networking time, and most importantly the finer
level breakdown of the computation time to understand the bottleneck of computation on mobile
devices. 

We categorized computation time into four categories: scripting, loading, rendering and
painting. Scripting is the total time spent on
compiling, evaluating and executing javascript. Loading consists of parsing the HTML and CSS, which happens 
immediately after the payload for the network requests are received by the browser. Loading
takes these payload objects and parses them before converting them to a DOM tree. Once the DOM tree is built,
the rendering engine converts this DOM tree into a render tree, which contains the
exact coordinates and the shape of each of the DOM nodes. This process comprises the rendering time of the web page.
Painting time is the time taken to process the render tree and convert
each pixel into a bitmap.
Figure 6 shows the computation breakdown for these four categories
on the Google Pixel 2. We further break down this time into the finer level events
which are returned by Google Chrome's trace and then group them by their event name.

\begin{figure}[t!]
\centering
\includegraphics[width=0.9\columnwidth]{figs/comp_2.png}
\tightcaption{Breakdown of computation into finer events on pixel 2}
\label{fig:cat_p2}
\end{figure}


The results in Figure 7 show the promising impact a Javascript 
caching mechanism would have on the total page load time.

\subsection{Current Chrome Optimizations}

\begin{figure}[t]
\centering
\includegraphics[width=0.9\columnwidth]{figs/chrome_script.png}
\tightcaption{CDF of scripting time with and without chrome's optimizations}
\label{fig:scripting_p2}
\end{figure}

Recently in their 2017 dev summit, the Chrome team discussed the various optimization techniques
they have developed to improve the total page load time.
We did a comparison of the total page load time with and without Chrome's optimizations to study
these improvements. We captured the trace from Alexa's top 75 
news and sports website once with a fresh cache, i.e. cold cache, and then subsequently with a hot
cache which contains all of Chrome's optimizations, including its compiler and parser cache. 
As seen in Figure 10, there has been a significant reduction in the overall compilation
time, with about 100ms reduction in median compile time. This is primarily due to the introduction of the compiler and parser cache. 
The line corresponding to cold cache refers to the fresh load of all the websites,
whereas the line corresponding to the hot cache refers to the subsequent load
which makes use of Chrome's caching framework. 
This is also reflected partially in the overall scripting time
as shown in Figure 8. Note that scripting time is the sum of compilation, execution and other
minor javascript events in the execution pipeline such as garbage collection. 
However, the interesting thing to note is that despite all these optimizations,
we observe almost negligible improvement in the median execution time of the Javascript, as
shown in Figure 9. This serves as motivation for the vast potential a caching framework would have in
improving the overall page load time.

\begin{figure}[t]
\centering
\includegraphics[width=0.9\columnwidth]{figs/chrome_exec.png}
\tightcaption{CDF of total execution time with and without chrome's optimizations}
\label{fig:compile_p2}
\end{figure}

\begin{figure}[t]
\centering
\includegraphics[width=0.9\columnwidth]{figs/chrome_compile.png}
\tightcaption{CDF of compilation time with and without chrome's optimizations}
\label{fig:compile_p2}
\end{figure}

\subsection{Experiments Conducted}

After establishing the impact of a Javascript execution caching framework, we conducted
experiments to understand Javascript computation at a finer granularity. We have explained the
results from these experiments in section 3.1. 
We observed that most of the properties of the global window object remain unchanged.
For a time difference of three seconds, we observe that only 2.5\% of properties
changed as shown in Figure 3. This is to be expected since little will change within three seconds 
of two subsequent web page loads. Surprisingly, even for a gap of three hours between
two loads, only 3.5\% of the properties changed as shown in Figure 4 and for a gap
 of three days, only 4.5\% 
properties changed. These are the 95\% percentile numbers, and therefore 
further motivate us to expect extremely high gains from a Javascript caching
framework. 

\begin{figure}[t]
\centering
\includegraphics[width=0.9\columnwidth]{figs/cdf_bigdata_sec_new.pdf}
\tightcaption{Percentage of changed properties over 3 seconds}
\label{fig:properties-sec}
\end{figure}

\begin{figure}[t]
\centering
\includegraphics[width=0.9\columnwidth]{figs/cdf_bigdata_hr_new.pdf}
\tightcaption{Percentage of changed properties over 3 hours}
\label{fig:properties-hrs}
\end{figure}

\begin{figure}[t]
\centering
\includegraphics[width=0.9\columnwidth]{figs/cdf_bigdata_day_new.pdf}
\tightcaption{Percentage of changed properties over 3 days}
\label{fig:properties-day}
\end{figure}


Currently, we are working on capturing the Javascript execution at the function level. 
In order to do this, we have built
a web proxy which sits between the client browser and the news and sports websites. Every time 
a request is made by the client, the proxy intercepts the request, injects instrumentation
code in the javascript files, and injects inline script tags inside the HTML files. 
This instrumented code is read by the Javascript debugger when the page is loaded, and 
the debugger then builds a call graph, with each node representing a function that was invoked. 
Once a graph is built, we will use a graph diffing algorithm to quantify
how much of the call graph was modified across the two loads. 


\section{Related Work}
\label{sec:related}

Work on improving web performance has been going for more than two decades now. 
Prior work has focused on various components of the overall page load time
from remodifying the source code of the webpage itself, to optimizing the
network component of the over all execution time and more recently some work
has been in improving the computatio time latency. 

Erman et all \cite{erman2013conext} %In Proceedings of the Ninth ACM Conference on Emerging Networking Experiments and Technologies
has shown that unline desktop browsers, optimizations such as SPDY/HTTP@ does not improve
performance of Web pages on mobile browsers. They show that this is because of the negative interactions
between the cellular state machine and the transport protocol. SImilary Qian
et al \cite{quian2012mobisys} %Web caching on smartphones: Ideal vs. reality. In MobiSys 
show that caching does not provide
page load improvements for mobile browsers. 
Google already released a paper last year \cite{vesuna2016caching} %Caching Doesn’t Improve Mobile Web Performance (Much)
which talks of how there is very little improvement to the overall page load time despite 
significant improvements in the caching hit rate. 

Many of the research on explicitly improving mobile browser performance has seen mixed results.
FLywheel \cite{agababov2015nsdi} is Google's compression proxy that compressses web content to significanlty 
reduce the use of expensive cellular data. The authots note that whie Flywheel succeeds in
reducong the data usage, its effect on page load performance is more mixed; it helps
performance of certain pages and hurts performance of others. Flexiweb \cite{singh2015mobicom} is built over
Google's compression proxy to ensure that the proxy does not hurt page load times. But FlexiWeb
is not designed to explicitly improve page load performance. Wang et al \cite{wang2013demystifying} show that
speculative loading in one of only client only approaches that can improve mobile
browser performance. However, speculative loading requires knowledge of what objects are likely
to be requested by the user. 

Other research works have looked at metrics orthognal to the page load time metric. Parcel
\cite{sivakumar2014conext} %Proxy assisted browsing in cellular networks for energy and latency reduction.
 uses a proxy
approach to divide the page load process between the mobile device and the proxy. Because Parcel is
a network appraoch, the evaluations are largely with respect to reduction to network
latencies. Klotski \cite{butkiewicz2015usenix} focusses on increasing the number of objects rendered in the first
5 seconds to improve the user quality of experience. 

Other client side imporvements reduce enrge usage and computational delays using parallel
browsers \cite{mai2012hotpar, meyerovich2010www} %A Case for Parallelizing Web Pages. In HotPar,  Fast and Parallel Web Page Layout. In WWW.
and improved hardware \cite{zhu2013hpca} % High-Performance and Energy-Efficient Mobile Web Browsing on Big/Little Systems. In HPCA
 By improving the parallelization 
for necessary page load tasks (eg: rendering) these systems reduce energy usage
and have positive impact on page load times. 

While there has been several recent efforts on improving mobile browser performance
they have not been uniformly successful.


\section{Future work}
\label{sec:future-work}

Most of the project up till now has been to serve as a motivation for building 
a caching framework for the Javascript execution output. All of our experiments
have shown the positive benefits of a caching framework. 
We also have established an upper bound on the posible decrease in the page load time
in the best case scenario. 

The actual implementation of the caching framework will be our next step. 
This is more like an engineering effort, which will determine the efficacy
of our idea when we can fully evaluate the sytem. 
In order to do this, we will have to modify the production level 
source code of the Chrome browser. 

Our understanding of the current caching framework for compiler
and parser cache can be used as a reference for our Javascript caching
framework. 


\section{Contribution}
\label{sec:contribution}

Our work had a fair distribution among the three co authors. 
This being the main reseach project of Ayush Goel, he was primarily 
responsible for setting up the testbed for the experiments and conducting them. 
All the code for the different anaylsis done for the page loads, like network
analysis, trace analysis and capturing the window object state and running
a simple diff algorithm is written by Ayush Goel. He has studied the current 
optimisations in place done by Chrome, and evaluated the improvement in the
loading time with these optimisations enabled. 

Matthew Furlong, the second co author has contributed in finding 
the javascript libraries we currently use to capture the network
and timeline trace for each web page load, on top of which most of 
our other analysis code is written. He has contributed in analysing
data from our experiments by building CDFs of the results and 
comparing them with each other. 

Hyunjong Lee, the third co author has contributed in writing the mid semester
and the final reports, generating figures and writing the content. 

Most importantly, Matthew and Hyunjong were constantly involved in the
critical discussion of the key ideas, and helped in estabilishing the next
steps for our research project. 
\label{lastpage}

\clearpage
\bibliographystyle{acm}
\bibliography{refs}

\clearpage
\appendix

\end{document}

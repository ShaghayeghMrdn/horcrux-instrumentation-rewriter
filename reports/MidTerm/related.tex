\section{Related Work}
\label{sec:related}

Many prior works~\cite{erman2013conext, quian2012mobisys,
sivakumar2014conext, mai2012hotpar, meyerovich2010www, zhu2013hpca}
in the domain of improving web performance have been
done, ranging from optimizing webpage to reducing network time.
Recent works~\cite{vesuna2016caching, agababov2015nsdi,
butkiewicz2015usenix} propose improving computation time to further
reduce PLT.

Recent studies~\cite{erman2013conext, quian2012mobisys,
vesuna2016caching} show that computation time is major component in
PLT of mobile browsers.  Erman et al.~\cite{erman2013conext} has shown
that unlike desktop browsers, optimizations such as SPDY/HTTP2 do not
improve PLT on mobile browsers. The authors show that this is because
of the negative interactions between the cellular state machine and
the transport protocol.  Similar to our study, Qian et
al.~\cite{quian2012mobisys} argue that caching resources does not
contriute to reduction in PLT for mobile devices. Furthermore, A
recent study~\cite{vesuna2016caching} shows that how there is very
little improvement to the overall page load time despite significant
improvements in the cache hit rate. 

On contrast to, the research on explicitly improving mobile browser
performance has seen mixed results.
FLywheel~\cite{agababov2015nsdi} is Google's compression proxy that compresses web content to significantly 
reduce the use of expensive cellular data. The authors note that while Flywheel succeeds in
reducing the data usage, its effect on page load time is more mixed; it helps
the performance of certain pages and hurts the performance of others.
Flexiweb~\cite{singh2015mobicom} is built over
Google's compression proxy to ensure that the proxy does not hurt page load times. However, FlexiWeb
is not designed to explicitly improve page load performance. Wang et
al.~\cite{wang2013demystifying} show that
speculative loading in one of only client only approaches that can improve mobile
browser performance. However, speculative loading requires knowledge of what objects are likely
to be requested by the user. 


Other research works have looked at metrics that are orthogonal to the
page load time metric. Parcel~\cite{sivakumar2014conext} uses a proxy
approach to divide the page load process between the mobile device and
the proxy. Because Parcel is a network approach, the evaluations are
primarily focused on the reduction of network latency.
Klotski~\cite{butkiewicz2015usenix} focuses on increasing the number
of objects rendered in the first 5 seconds to improve the user quality
of experience. 

Other client side improvements reduce energy usage and computational
delays using parallel browsers~\cite{mai2012hotpar, meyerovich2010www}
and improved hardware~\cite{zhu2013hpca}.  By improving the
parallelization for necessary page load tasks, such as rendering,
these systems reduce energy usage and have a positive impact on page
load times. 

% While there have been several recent efforts on improving mobile browser performance
% they have not been uniformly successful due to their various limitations.

